
\usepackage{booktabs,tikz, ifthen, amsmath,dcolumn}
\usepackage{xcolor}

%\usetikzlibrary{calc}
\usetikzlibrary{
	decorations, 
	decorations.pathmorphing,
	decorations.text,
	patterns,
	backgrounds,
	arrows}

%:-----------------------------------------------------------------
%:new definitions
%:-----------------------------------------------------------------
\def\lw{\linewidth}
\def\tw{\textwidth}
\def\th{\textheight}
\def\U{\unit}

\def\kC{\dfrac{1}{4 \pi \epsilon_0}}

%:-----------------------------------------------------------------
%:colours
%:-----------------------------------------------------------------
%\definecolor{tufte}{RGB}{255,255,248}
\definecolor{tufte}{RGB}{255,255,255}
\definecolor{Maroon}{RGB}{139, 0, 0}
%:-----------------------------------------------------------------
%:new environment
%:-----------------------------------------------------------------

%http://tex.stackexchange.com/questions/39296/simulating-hand-drawn-lines
\pgfdeclaredecoration{free hand}{start}
{
	\state{start}[width = +0pt,
	next state=step,
	persistent precomputation = \pgfdecoratepathhascornerstrue]{}
	\state{step}[auto end on length    = 3pt,
	auto corner on length = 3pt,               
	width=+2pt]
	{
		\pgfpathlineto{
			\pgfpointadd
			{\pgfpoint{2pt}{0pt}}
			%      {\pgfpoint{rand*0.15pt}{rand*0.15pt}}
			{\pgfpoint{rand*0.15pt}{rand*0.15pt}}
		}
	}
	\state{final}
	{}
}
\tikzset{free hand/.style={
		decorate,
		decoration={free hand}
	}
}

%:myPic
\newenvironment{myTikz}[2][]{
	\def\w{\linewidth}
	\def\h{#2}
	\begin{tikzpicture}
		[
		background rectangle/.style={fill=tufte},
		 show background rectangle,
		>=stealth,%		 >= triangle 60,		 
		 font=\large, 
		  thick
		 ]
		%	Set the bounding box
%		\useasboundingbox (0,0) rectangle +(\w,\h+0.01*\h);

}{\end{tikzpicture}}

\def\freedraw#1;{\draw[free hand] #1;}

%:-----------------------------------------------------------------
%:new functions
%:-----------------------------------------------------------------
\newcommand{\myvec}[4][ultra thick]{
	%options, start, end-angle, end-length
	\freedraw[{#1}]  {#2} -- +({#3}:{#4});
	\freedraw[-triangle 45,{#1}]  {#2} ++({#3}:{.4*#4})-- +({#3}:{.1*#4});
}

%:modifies table 'headers'
\newcolumntype{C}[1]{>{\centering\let\newline\\\arraybackslash\hspace{0pt}}m{#1}}
\newcolumntype{L}[1]{>{\raggedright\let\newline\\\arraybackslash\hspace{0pt}}m{#1}}
\newcolumntype{R}[1]{>{\raggedleft\let\newline\\\arraybackslash\hspace{0pt}}m{#1}}
